\documentclass[12pt]{article}
\usepackage[utf8]{inputenc}
\usepackage[T1]{fontenc}
\usepackage{cite}
\usepackage{graphicx}
\usepackage{float}
\usepackage{multirow}
\usepackage[table,xcdraw]{xcolor}
\usepackage[colorlinks]{hyperref}
\usepackage{epstopdf}
\hypersetup{
  colorlinks = true,
  linkcolor  = magenta,
  citecolor  = magenta
}
\usepackage[all]{hypcap}
\usepackage{listings}
\usepackage{color}
\definecolor{dkgreen}{rgb}{0,0.6,0}
\definecolor{gray}{rgb}{0.5,0.5,0.5}
\definecolor{mauve}{rgb}{0.58,0,0.82}
\lstset{frame=tb,
  language=Python,
  aboveskip=3mm,
  belowskip=3mm,
  showstringspaces=false,
  columns=flexible,
  basicstyle={\small\ttfamily},
  numbers=none,
  numberstyle=\tiny\color{gray},
  keywordstyle=\color{blue},
  commentstyle=\color{dkgreen},
  stringstyle=\color{mauve},
  breaklines=true,
  breakatwhitespace=true,
  tabsize=3
}

\title{Models for Sustainable \\Population Growth on Mars}
\author{Chris Dunlap, Allen Koh, Matt May}
\date{CSE 6730 Project 2 Checkpoint, Spring 2016}

\begin{document}

\begin{titlepage}
  \maketitle
  \thispagestyle{empty}
\end{titlepage}

\newpage
  \tableofcontents
  \thispagestyle{empty}
\newpage

\section{Problem Statement and Approach}
\label{sec:problem}

In the United States, the National Aeronautics and Space Administration (NASA)
has announced its plan to send humans to Mars during the 2030s. This ambitious
goal requires a variety of studies be conducted to effectively plan the
endeavor. General habitation, food production, resource extraction,
communication, spacecraft, and many other areas must be studied to determine
their optimal configuration.

For our second project, we propose a simulation of population growth dynamics on
Mars, with the goal of determining an optimal strategy for sustainable
population growth. Population growth models have been extensively studied in the
literature \cite{clark1967population}, \cite{caswell2001matrix},
\cite{meadows1992beyond}, \cite{boserup1983population},
\cite{ehrlich1971impact} but generally only in the context of our own planet.

\section{Simulation Description}
\label{sec:simdescrip}

We intend to model humans as consumer entities, and several
types of resources such as food, water, and sanitation availability as resource
entities. We intend to take a stochastic, discrete-time approach. As David
Quammen notes \cite{quammen1996song}, there are four sources of uncertainty to
which a population may be subject: demographic, environmental, natural
catastrophes, and genetic. We will attempt to model several of these to
provide the greatest realism possible.

\section{Related Work}
\label{sec:relatedwork}

Often, natural populations without resource limitations exhibit exponential
growth \cite{audesirk1996biology}. However, this type of rapid growth will
likely be unsustainable under the extreme resource constraints of Mars.
By considering several proposed habitation models for Mars, we hope to better
understand the resource requirements of these approaches, and by that develop
recommendations for sustainable growth.

A stochastic model of population growth	during the Neolithic transition focused
on foragers and farmers is presented by \cite{fedotov2008stochastic} where a
two-population model is used. Foragers and farmers are modeled separately but
maintain a relationship through total population density. Crop production is
also modeled by a formula based on soil nutrients and production rate.  The
density of soil nutrients is modeled as a partial differential equation, taking
into account population size and crop production per unit of time.  The study
discusses the change in food supply as population density increases and farm
land degrades, but as mentioned earlier, there are underlying assumptions that
do not apply to the case of colonizing Mars (such as erosion and flooding).
Despite the model's end goal being different from ours, the modeling of crop
production and soil nutrients appears transferrable to our application with
the proper tailoring.

The study in \cite{fedotov2008stochastic} established the use of phosphorus as
the predominant indicator of nutrients in soil.  As such, the relationship of
phosphorus excreted by human subjects as a function of protein intake
\cite{pooOnYOu} can be applied to our problem to quantify
the ability to reconstitute soil for farming by using human excrement.

A study by the Food and Agriculture Organization of the United Nations (FAO)
\cite{faoProtein} further adds value to our study by providing the
amount of animal- and plant-based protein consumed by individuals from a
multitude of contries.  Given our study's focus on NASA, figures from the United
States can be gleaned.  Other information from the FAO \cite{faoNutrition}
provides figures for crop efficiency, quantifying edible energy
and protein per hectacre of farming land for a selection of key crops.

Finally, a study related to hydroponics \cite{iHeartHydroponics} presents a
final useful component to the modeling of food our study, where the effectiveness
of hydroponic gardening is compared to that of conventional crop growing techniques
by analyzing one of the crops found in \cite{faoNutrition}.
In the end of the study, a multiplier is found that could be used to approximate
the amount of food the Mars colony can grow using hydroponics when compared to
the amount grown by conventional means.  In this light,
it may be attractive for the Mars colony to use hydroponics in lieu of
conventional farming landscapes.

Another study by \cite{moore2001evaluating} introduces five models of human
colonization.  The study focuses around expansion of colonies by modeling
migration patterns of the population as well as mortality and fertility rates.
The five models of colonization mentioned are the matrix model, beachhead
model, string of pearls, outpost model and the pulse model. The paper
concludes that regardless of population size, low fertility rates and/or high
mortality rates will cause colonization to fail.

\section{Simulation Architecture}
\label{sec:architecture}

\section{Progress To Date}
\label{sec:progress}

From a programming perspective, we plan to use the Python programming language,
which is object-oriented, dynamically typed, and interpreted, making it an
excellent choice for developing our simulation in an iterative manner.

\section{Task Plan}
\label{sec:taskplan}

\section{Proposal}
\label{sec:proposal}

\clearpage
\bibliography{template}{}
\bibliographystyle{plain}
\end{document}
