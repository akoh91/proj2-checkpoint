\documentclass[12pt]{article}
\usepackage[utf8]{inputenc}
\usepackage[T1]{fontenc}
\usepackage{cite}
\usepackage{graphicx}
\usepackage{float}
\usepackage{multirow}
\usepackage[table,xcdraw]{xcolor}
\usepackage[colorlinks]{hyperref}
\usepackage{epstopdf}
\hypersetup{
  colorlinks = true,
  linkcolor  = magenta,
  citecolor  = magenta
}
\usepackage[all]{hypcap}
\usepackage{listings}
\usepackage{color}
\definecolor{dkgreen}{rgb}{0,0.6,0}
\definecolor{gray}{rgb}{0.5,0.5,0.5}
\definecolor{mauve}{rgb}{0.58,0,0.82}
\lstset{frame=tb,
  language=Python,
  aboveskip=3mm,
  belowskip=3mm,
  showstringspaces=false,
  columns=flexible,
  basicstyle={\small\ttfamily},
  numbers=none,
  numberstyle=\tiny\color{gray},
  keywordstyle=\color{blue},
  commentstyle=\color{dkgreen},
  stringstyle=\color{mauve},
  breaklines=true,
  breakatwhitespace=true,
  tabsize=3
}

\title{Models for Sustainable \\Population Growth on Mars}
\author{Chris Dunlap, Allen Koh, Matt May}
\date{CSE 6730 Project 2 Checkpoint, Spring 2016}

\begin{document}

\begin{titlepage}
  \maketitle
  \thispagestyle{empty}
\end{titlepage}

\newpage
  \tableofcontents
  \thispagestyle{empty}
\newpage

% The problem and questions you are addressing and approach you will use
\section{Problem Statement and Approach}
\label{sec:problem}

In the United States, the National Aeronautics and Space Administration (NASA)
has announced its plan to send humans to Mars during the 2030s. This ambitious
goal requires a variety of studies be conducted to effectively plan the
endeavor. General habitation, food production, resource extraction,
communication, spacecraft, and many other areas must be studied to determine
their optimal configuration.

For our second project, we propose a simulation of population growth dynamics on
Mars, with the goal of determining an optimal strategy for sustainable
population growth. Population growth models have been extensively studied in the
literature \cite{clark1967population}, \cite{caswell2001matrix},
\cite{meadows1992beyond}, \cite{boserup1983population},
\cite{ehrlich1971impact} but generally only in the context of our own planet.

Often, natural populations without resource limitations exhibit exponential
growth \cite{audesirk1996biology}. However, this type of rapid growth will
likely be unsustainable under the extreme resource constraints of Mars.
By considering several proposed habitation models for Mars, and modeling the
effects of uncertainty, we hope to better understand the viability of these
approaches, and by that develop recommendations for sustainable growth.

% A description of the simulation you are developing
\section{Simulation Description}
\label{sec:simdescrip}

We intend to model humans as consumer entities, and several
types of resources such as food, water, and sanitation availability as resource
entities. We intend to take a stochastic, discrete-time approach. As David
Quammen notes \cite{quammen1996song}, there are four sources of uncertainty to
which a population may be subject: demographic, environmental, natural
catastrophes, and genetic. We will attempt to model several of these to
provide the greatest realism possible.

\subsection{Inputs}
\subsection{Outputs}
\subsection{Content}
\subsection{Assumptions \& Simplifications}

\section{Related Work}
\label{sec:relatedwork}

\subsection{Population Growth Models}

Moore \cite{moore2001evaluating} introduces five models of human
colonization. The study focuses around expansion of colonies by modeling
migration patterns of the population as well as mortality and fertility rates.
The five models of colonization mentioned are the matrix model, beachhead
model, string of pearls, outpost model and the pulse model. The paper
concludes that regardless of population size, low fertility rates and/or high
mortality rates will cause colonization to fail.

There has been previous literature on modeling population growth with limited
resources. The most popular model used for population growth is the matrix
model.  Miller et al \cite{miller2002density} use a Leslie matrix
model to estimate the annual increase of a gray wolf population. This model
takes inputs of survival and fertility rates and is modified for an environment
with limited resources. A simple density-dependent matrix model is used based
on a discrete time scalar logistic equation with a defined carrying capacity
factor. The estimates for the projection matrix, including survival rates,
fertility rates, litter size, and carrying capacity, were taken from field
studies of real populations. Aikio \& Pakkasmaa present additional
characteristics to model population growth by linking growth and reproduction
rates with an individual's biomass and the number of individuals they
interact with \cite{aikio2003relatedness}. Clark \& Innis uses a model that
integrates energy and protein relationships for jack rabbit population growth
where the limited resource is food \cite{clark1982forage}. Food intake is
controlled by energy balance and gut fill while foraging selection is used to
balance nutrients. The growth and reproduction rates have energy and protein
requirements while mortality rates are influenced by predation from coyotes and
natural causes.  Peterson et al \cite{peterson2014modeling} also introduces a
matrix model structure around population dynamics of trees in a forest and
mentions the ease of computer simulation as a factor behind using the matrix
model structure. The forest matrix model predicts population dynamics using
vectors of live trees as well as growth and recruitment matrices.

\subsection{Food Production}

A stochastic model of population growth	during the Neolithic transition focused
on foragers and farmers is presented by Fedotov \cite{fedotov2008stochastic}. A
two-population model is used in which foragers and farmers are modeled
separately but maintain a relationship through total population density.
Crop production is also modeled by a formula based on soil nutrients and
production rate. The density of soil nutrients is modeled as a partial
differential equation, taking into account population size and crop production
per unit of time. The study discusses the change in food supply as population
density increases and farm land degrades; however, there are underlying
assumptions that do likely not apply to the case of colonizing Mars (such as
erosion and flooding). Despite the model's objectives being different from our
own, the modeling of crop production and soil nutrients appears transferrable
to our application with the proper tailoring.

Fedotov's study \cite{fedotov2008stochastic} suggests the use of phosphorus as
the predominant indicator of nutrients in soil. As such, the relationship of
phosphorus excreted by human subjects as a function of protein intake
\cite{pooOnYOu} can be applied to our problem to quantify the ability to
reconstitute soil for farming by using human excrement.

A study by the Food and Agriculture Organization of the United Nations (FAO)
\cite{faoProtein} further adds value to our study by providing the
amount of animal and plant-based protein consumed by individuals from a
multitude of contries. Given our study's focus on NASA, figures from the United
States can be gleaned. Other information from the FAO \cite{faoNutrition}
provides figures for crop efficiency, quantifying edible energy
and protein per hectare of farming land for a selection of key crops.

Finally, a study related to hydroponics \cite{iHeartHydroponics} presents a
final useful component to the modeling of food our study, where the
effectiveness of hydroponic gardening is compared to that of conventional crop
growing techniques by analyzing one of the crops found in \cite{faoNutrition}.
At the conclusion of the study, a multiplier is noted that could be used to
approximate the amount of food the Mars colony can grow using hydroponics when
compared to the amount grown by conventional means. In this light, it may be
attractive for the Mars colony to use hydroponics in lieu of conventional
farming techniques.

\subsection{Mars-Specific Habitation Models}

\subsubsection{Oxygen Generation}

The distance from Earth to Mars varies between roughly 58 million and 400
million km. Because of the great distance, resupply capbilities for human
missions to Mars are virtually nonexistent. Due to this challenge, innovative
approaches must be taken to regenerate necessary resources in-situ. To this end,
the Mars Design Reference Architecture 5.0 \cite{drake2010human} proposes
the use of an ``In-Situ Resource Utilization System'' (ISRU) that converts Mars
atmosphere into oxygen for both propellant and life support purposes. The plant
operates by using electrolyzers that convert carbon dioxide into oxygen and
carbon monoxide, which is then vented. A hydrogen feedstock is brought from
Earth and reacted with Mars-produced oxygen to generate water. Furthermore,
carbon dioxide, nitrogen, and argon that are extracted from the atmosphere of
Mars can be employed as a buffer gas for crew breathing.

\subsubsection{Power Generation}

The lack of known resources on Mars that can be mined for power generation
requires either a solar-based power source or a power source transported from
Earth \cite{hoffman1997human}. As noted, Mars is roughly 50\% farther from the
Sun on average relative to the Earth, so only around half of the solar
radiation that Earth experiences actually reaches Mars. Thus, the Design
Reference Architecture \cite{drake2010human} proposes the use of a nuclear
fission-based reactor. Of the power sources which can be transported from
Earth, a nuclear power source is the only known option that concentrates
sufficient energy in a reasonable mass and volume \cite{hoffman1997human}.
The reactor operates at a low temperature, allowing stainless steel (which is
compatible with Mars' atmosphere) to be used for reactor components. The ISRU
plant mentioned previously is a predominant consumer of power (consuming 25 kWe
when operating continuously).

\subsubsection{Space Agriculture and Waste Processing}

The daily resource requirements of humans have been studied through computer
simulation \cite{rummel1987modular}. Simulations indicate that a human requires
(per day) 855g of food inputs, 4577g of drinking/food preparation water, 128g
of water in food, 18,000g of wash/flush water, and 804g of oxygen for food
metabolism. In terms of outputs per day, they are separated into three
categories: water, solids, and carbon dioxide. Humans produce 3025g of water
in urine and feces, 406g of metabolic water (vapor), 1680g of perspiration
water (vapor), and 18,000g of wash/flush water. 161g of solids in the form of
feces, urine, and sweat solids are produced. 1092g of carbon dioxide from
food metabolization is produced. Over the course of a year, this means that
the average human is consuming roughly ``three times his body weight in food,
four times his weight in oxygen, and eight times his weight in drinking water.''
\cite{modell1980rationale}. Thus, bioregenerative systems are essential to
sustainable long-term habitation on Mars.

To enable space agriculture, hyper-thermophilic aerobic composting bacteria
have been studied specifically in the context of habitation on Mars
\cite{kanazawa2008space}. This technology can be implemented as a subsystem
that oxidizes inedible biomass/wastes, converting them to fertilizer. High
quality compost is an important part of a regenerative food production system,
but as noted, must be implemented carefully to avoid the spread of pathogenic
bacteria. A ``marsh-based waste processing system'' \cite{nelson1992biosphere}
has also been studied which exploits the natural ability of aquatic plant/
microbial associations to perform processing of waste. These systems metabolize,
or concentrate, pollutants while generating useful biomass growth. Furthermore,
aquatic plants can be used as purified water sources by condensing moisture
evapotranspired from plant leaves.

In another study \cite{katayama2005entomophagy}, a menu for a sustainable human
diet on Mars was developed, concluding that a combination of rice, soybeans,
sweet potatoes, green-yellow vegetables, silkworm pupa, and loach would fulfill
human nutritional requirements.

\section{Simulation Architecture}
\label{sec:architecture}

\section{Progress To Date}
\label{sec:progress}

From a programming perspective, we plan to use the Python programming language,
which is object-oriented, dynamically typed, and interpreted, making it an
excellent choice for developing our simulation in an iterative manner.

\section{Task Plan}
\label{sec:taskplan}

Going forward, tasks for for our team will be divided as follows:

\begin{itemize}
  \item Chris will be focusing on developing
  \item Allen will focus on
  \item Matt's focus is
\end{itemize}

\clearpage
\bibliography{template}{}
\bibliographystyle{plain}
\end{document}
